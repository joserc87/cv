% José Ramón Cano's CV

% --------------
% MULTI LANGUAGE
% --------------

% Make this document multilanguage by using ifthen
% There should be a tex document calling this master document, defining the
% output language, i.e.:
% \def\outputlanguage{en}
%
% SUPPORTED LANGUAGES: [en, es]
% Credits to @darkled
% (http://tex.stackexchange.com/questions/60781/managing-multiple-translation-of-a-single-document)

\RequirePackage{ifthen}
\newif\ifen{}
\newif\ifes{}
\newcommand{\en}[1]{\ifen#1\fi}
\newcommand{\es}[1]{\ifes#1\fi}
\newcommand{\ml}[2]{\ifen#1\fi\ifes#2\fi}

% If the output language is not defined, set it to english by default
\ifcsname outputlanguage\endcsname
\else
    \def\outputlanguage{en}
\fi
% Run \estrue or \entrue depending on the output langauge
\ifthenelse{\equal{\outputlanguage}{es}}
% IF outputlanguage is ES, output spanish
{\estrue{}}
% If outputlanguage is not ES, try EN
{\ifthenelse{\equal{\outputlanguage}{en}}
    % If outputlanguage is EN, output english
    {\entrue{}}
    % Here we could add more languages...
    % If language not supported, use english
    {\entrue{}}
}

% --------
% SETTINGS
% --------

\documentclass[11pt,a4paper,sans]{moderncv}   

% Temas de moderncv

% Options = ['casual'(default), 'classic']
\moderncvstyle{casual}                        
% Color = ['blue', 'orange', 'green', 'red', 'purple', 'grey', 'black']
\moderncvcolor{blue}                          

% sans serif as default font
%\renewcommand{\familydefault}{\sfdefault}
%\nopagenumbers{}
\usepackage[utf8]{inputenc}                   

\en{\usepackage[english]{}}
\es{\usepackage[spanish]{}}

% Margins
\usepackage[scale=0.75]{geometry}

% Words not known for hyphenation:
\hyphenation{assem-bler}

% To change the column width for dates
%\setlength{\hintscolumnwidth}{3cm}           

%----------------------
% PERSONAL INFORMATION
%----------------------

\firstname{José Ramón}
\familyname{Cano Yribarren}

\title{\ml{Software Engineer}{Ingeniero Informático}}

\address{Noordwal 68}{\ml
    % EN
    {The Hague, 2513EC (The Nederlands)}
    % ES
    {La Haya, 2513EC (Países Bajos)}
}

\mobile{+31~(0)6~51~03~04~66}
% \phone{+34~958~508~130}
% \fax{+3~(456)~789~012}
\email{joserc87@gmail.com}
% \homepage{http://joseramoncano.com}
% \extrainfo{Additional Information}

% PICTURE (optional):
% '64pt' is the height to which the image will be adjusted, and 0.4pt is the
% frame width containing it (set it to 0pt to remove the frame). 'picture' is
% the name of the fileis the height to which the image will be adjusted, and
% 0.4pt is the frame width containing it (set it to 0pt to remove the frame).
% 'picture' is the name of the file
% \photo[64pt][0.4pt]{picture}
% \quote{some optional quote}                            % optional

% To show numeric labels in the bibliography (by default they are not shown).
% Only useful when you add quotes in your CV
%\makeatletter
%\renewcommand*{\bibliographyitemlabel}{\@biblabel{\arabic{enumiv}}}
%\makeatother

% Bibliography with multiple sources
%\usepackage{multibib}
%\newcites{book,misc}{{Libros},{Otros}}

%---------
% CONTENT
%---------
\begin{document}

%\begin{CJK*}{UTF8}{gbsn}                     % para redactar el CV en chino usando CJK
\maketitle

\section{\ml{Personal information}{Información Personal}} 

\cvitem
    {\ml{First Name}{Nombre}}
    {José Ramón}

\cvitem
    {\ml{Last Name}{Apellidos}}
    {Cano Yribarren}

\cvitem
    {\ml{Gender}{Sexo}}
    {\ml{Male}{Varón}}

\cvitem
    {\ml{Nationality}{Nacionalidad}}
    {\ml{Spanish}{Española}}

\cvitem
    {\ml{Marital status}{Estado civil}}
    {\ml{Single}{Soltero}}

\cvitem
    {\ml{Age}{Edad}}
    {29}

\cvitem
    {\ml{Email}{Correo electrónico}}
    {joserc87@gmail.com}

%---------------------------------------------------------------------

\section{\ml{Languages}{Idiomas}}

\cvitemwithcomment
    {\ml{Spanish}{Español}}
    {\ml{Native}{Nativo}}
    {}

\cvitemwithcomment
    {\ml{English}{Inglés}}
    {\ml{Fluent}{Fluido}}
    {\ml{Work abroad for more than 4 years, speaking English in a daily
    basis.}{Trabajo en el extranjero durante más de 4 años, hablando inglés a
    diario.}}

\cvitemwithcomment
    {\ml{Dutch}{Neerlandés}}
    {\ml{Basic}{Inicial}}
    {}

%---------------------------------------------------------------------

\section{\ml{Experience}{Experiencia}}

\ml
    {\cventry{2012--Present}{Lead software engineer at R\&D}{The Consultancy Firm}{The Netherlands}{}{}}
    {\cventry{2012--Presente}{Ingeniero software, desarrollador principal en I+D}{The Consultancy Firm}{Países Bajos}{}{}}

\ml
    {\cventry{2010--2011}{Internship at Virtual Reality Lab}{University of Granada}{Spain}{}{}}
    {\cventry{2010--2011}{Beca en Laboratorio de Realidad Virtual}{Universidad de Granada}{España}{}{}}

%---------------------------------------------------------------------

\section{\ml{Education}{Educación}}

%\cventry{year}{Degree}{Institution}{City}{\textit{Grade}}{Description}
\ml
{\cventry{2005--2011}{Engineering in software and computer science}{University of Granada}{Spain}{}{
    \begin{itemize}
        \item Specialization in software engineering, AI and graphics.
        \item High grades
        \item Maximum grade (10) at final year project: implementation of a
            real time volumetric Ray-Casting renderer in GPU with support for
            stereoscopic vision.
    \end{itemize}
}} % English
{\cventry{2005--2011}{Ingeniería en informática}{Universidad de Granada}{España}{}{
    \begin{itemize}
        \item Especialización en ingeniería de software, IA y gráficos.
        \item Alta calificación global
        \item Maxima calificación (10) en proyecto fin the carrera
            (implementación en GPU de un visualizador en tiempo real de
            volúmens por Ray-Casting con visión estereoscópica.).
    \end{itemize}
}} % Spanish

\ml
    {\cventry{2003--2005}{Science high school diploma}{Instituto Padre Manjón}{Granada}{Spain}{}}
    {\cventry{2003--2005}{Bachiller científico}{Instituto Padre Manjón}{Granada}{España}{}}

\clearpage
%---------------------------------------------------------------------

\section{\ml{Knowledge}{Conocimientos}}

\cvitem{}{\ml
    % EN
    {My main areas of knowledge are the following (items are listed in
    descending order of experience):}
    % ES
    {Mis areas de conocimiento son principalmente las siguientes, por orden
    decreciente de experiencia:}}

\cvcomputer
    {\ml{Programming Languages}{Lenguajes de programación}}
        {C/C++, Java, C\#, Python, Objetive-C, Asm, PHP, JS, Ruby, Haskell.}
    {\ml{Technologies}{Tecnologías}}
        {XML (DOM, SAX, XPath, Serialization/Marshalling), WebServices
        (SOAP/REST), LDAP.}

\cvcomputer
    {Web Frameworks}
        {Ext.Net, MEAN stack, socket.io. \ml{Others}{Otros}: Laravel,
        Django, Ruby on Rails}
	{GUIs}
        {Swing, Cocoa Touch, Qt, SDL.}

%--

\cvcomputer
    {\ml{Software Design}{Diseño de Software}}
        {\ml
            % EN
            {UML, Object Orientation, MVC, SOLID principles Design Patterns,
            TDD/BDD, data structures and algorithms, CVS (git/svn).}
            % ES
            {UML, Orientación a Objetos, MVC, SOLID principles Patrones de
            Diseño, TDD/BDD, estructuras de datos y algoritmos, CVS (git/svn).}
        }
    {\ml{Testing Libraries}{Bibliotecas de unit testing}}
        {Xunit, JUnit, Mocha, etc}


%--

\cvcomputer
    {OS}
        {\ml
            % EN
            {Advanced use of GNU/Linux, *nix CLI (e.g. Mac OS X) and kernel
            compiation. Windows at user level and basics of Windows Sever and
            DOS scripting.}
            % ES
            {Uso avanzado de GNU/Linux, *nix CLI (e.g. Mac OS X) y compilación
            del núcleo. Windows a nivel usuario y conocimientos básicos de
            Windows Server (IIS, etc) y scripting.}
        }
    {DB}
        {SQL/T-SQL, MicroSoft SQL Server, MySQL, Oracle.}

%--

\section{\ml{Others}{Otros}}

%--

\cvcomputer
    {\ml
    % EN
    {Background in Artificial Intelligence and Soft Computing}
    % ES
    {Nociones de Inteligencia Artificial y Soft Computing}}
        {\ml
        % EN
        {Bio-inspired and evolutionary algorithms, Artificial Neural Networks
        (Matlab), Computer Vision (OpenCV), Pattern Recognition, System
        Simulation.}
        % ES
        {Algoritmos bio-inspirados y evolutivos, Red Neuronales Artificiales
        (Matlab), Visión por Ordenador (OpenCV), Reconocimiento de Patrones,
        Simulación de Sistemas.}}
    {\ml
    % EN
    {Background in Parallel Programming}
    % ES
    {Nociones de Paralelismo}}
        {\ml
        % EN
        {MPI, OpenMP, OpenCL, Threads.}
        % ES
        {Hebras, MPI, OpenMP, OpenCL.}}

%--

\cvcomputer
    {\ml{Electronics}{Electrónica}}
        {\ml
        % EN
        {Basic knowledge in electronics. Pic/AVR microcontrolers programming,
        just as a hobby. PLCs programming.}
        % ES
        {Conocimientos básicos de electrónica. Programacion de
        microcontroladores Pic/AVR/Arduino, solo como hobby. Programación de
        PLCs.}}
    {\ml{Computer Graphics}{Gráficos por ordenador}}
        {\ml
        % EN
        {OpenGL (VBO, DL, FBO, GLSL), raytracing, volumetric-raycasting.}
        % ES
        {OpenGL (VBO, DL, FBO, GLSL), raytracing, raycasting volumétrico.}}

%--

\cvcomputer
    {\ml{Favorite IDEs}{IDEs favoritos}}
        {Vim + Tmux!, VisualStudio, IntelliJ IDEA, Eclipse, Netbeans,
        QtCreator}
    {\ml{Other Software}{Otro Software}}
        {Emacs, Git, SVN?, Makefile, PlantUML, Doxygen, GNU-plot}

\cvcomputer
    {\ml{Office Suite}{Ofimática}}
        {\LaTeX, Sphynx, Markdown, \ml{and when needed}{y si es
        necesario} OpenOffice/Microsoft Office.}
    {\ml{Other Skills}{Otras Habilidades}}
        {\ml
        % EN
        {Problem Solving. Mathematical/Algorithmic and Abstract thinking. Good
        communication skills and team work. Ease of learning \& curiosity.
        Perfeccionist.}
        % ES
        {Resolución de Problemas. Pensamiento Matemático/Algorítmico y
        Abstracto. Buenas habilidades de comunicación y trabajo en equipo.
        Facilidad para el aprendizaje y naturaleza curiosa.  Perfeccionista.}}

\clearpage
%-------------------------------------------------------------------------------

\section{\ml{Projects @ The Consultancy Firm}{Proyectos @ The Consultancy Firm}}

%--

\cvitem{The Document Wizard}
    {\ml
    % EN
    {Main project. A web application written in C\# (backend) using ASP.Net,
    Ext.Net (frontend), MicroSoft SQL Server (persistance), ActiveDirectory
    (authentication), WCF (web services), etc. This application connects to
    \textit{OpenText}'s \textit{Stream Server} via it's SOAP API to generate
    documents based on answers to questions pre-programmed by the user.}
    % ES
    {Proyecto principal. Una aplicación web escrita en C\# (backend) y usando
    ASP.Net, Ext.Net (frontend), MicroSoft SQL Server (persistencia),
    ActiveDirectory (autenticación), WCF (web services), etc. Esta aplicación
    se conecta con el producto de \textit{OpenText}, \textit{Stream Server} via
    su servicio SOAP para generar documentos PDF basados en respuestas a
    preguntas pre-programadas por el usuario.}
}

%--

\cvitem{DocWiz}
    {\ml
    % EN
    {An iOS app written in Objective-C that works as an online or offline client
    for The Document Wizard.}
    % ES
    {Una app para iOS escrita en Objective-C que hace las veces de cliente
    (online o offline) para \textit{The Document Wizard}.}
}

%--

\cvitem{Augmented Documents}
    {\ml
    % EN
    {An augmented reality app for iOS using OpenGL, and MetaIO SDK. (proof of concept)}
    % ES
    {Una app de realidad aumentada para iOS usando OpenGL y el SDK de MetaIO. (proof of concept)}
}

%--

\cvitem{The Apprentice}
    {\ml
    % EN
    {An intuitive WYSIWYG editor written in Java, with Swing for the GUI, to
    generate "Question" configurations for The Document Wizard.  Those
    configurations are marshalled to a file or directly uploaded with SOAP.}
    % ES
    {Editor WYSIWYG escrito en Java and Swing para generar archivos de
    configuracion de "Preguntas" para \textit{The Document Wizard}. Esta
    configuración es serializada (marshalled) en XML y enviada a través de
    SOAP.}
}

%--

\cvitem{Sandd Connector}
    {\ml
    % EN
    {A \textit{StreamServe} connector written in Java to sort the letters in the
    printing queue of a printshop based on a network file with the physical
    addresses in the Netherlands and according to a sorting and classification
    algorithm, so that the letters are produced in which Sandd would deliver
    them, minimizing the costs of posting.}
    % ES
    {Un conector para \textit{StreamServe} escrito en Java para ordenación de
    cartas de una cola de impresión de una imprenta, en base a una red con
    todas las direcciones de los países bajos y de acuerdo con unos algorimos y
    estandares de clasificación, de forma que la correspondencia es generada en
    el orden en la que el sistema de correos holandés (Sandd) las envía,
    minimizando los costes.}
}

%--

\cvitem{PDF connector}
    {\ml
    % EN
    {A \textit{StreamServe connector} written in Java and using the PDFBox
    library to fix `faulty' PDF output that \textit{StreamServe} generates.}
    % ES
    {Conector para \textit{StreamServe} escrito en Java y usando la biblioteca
    PDFBox para solucionar `problemas' en los archivos PDF generatods por
    este.}
}

%--

\cvitem{PS connector}
    {\ml
    % EN
    {A \textit{StreamServe connector} written in Java that parses
    \textit{PostScript} output to fix problems in its fonts.}
    % ES
    {Conector para \textit{StreamServe} escrito en Java que analiza archivos
    \textit{PostScript} y soluciona ciertos `problemas' relacionados con fuents
    de texto.}
}

%--

\cvitem{ASure}
    {\ml
    % EN
    {A notification connector in Java for error reporting to a MicroSoft SQL
    Server database using JDBC.}
    % ES
    {Conector Java para generar informes de error y rastreo en una base de
    datos Microsoft SQL Server.}
}

%--

\cvitem{SocioPath}
    {\ml
    % EN
    {A reactive web application written in \textit{NodeJS} using the
    \textit{MEAN} stack (\textit{MongoDB, ExpressJS, Angular 1, Socket.io,
    etc}) to integrate multiple social network communication channels (Facebook
    messenger, WhatsApp, Telegram and more) for customer service. It also acts
    as a messenger bot.}
    % ES
    {Aplicación web escrita en \textit{NodeJS} usando \textit{MEAN stack}
    (\textit{MongoDB, ExpressJS, Angular 1, Socket.io, etc}) que integra
    canales de comunicación de multiples redes sociales (Facebook messenger,
    WhatsApp, Telegram y más) para facilitar el servicio al cliente). También
    actua como un text-bot para messenger.}
}

%--

\section{\ml{Areas of interest}{Areas de interés}}

\cvlistdoubleitem
    {\ml{PC software development}          {Desarrollo de software para PC}}
    {\ml{Back end software development}    {Web backend/fullstack}}
\cvlistdoubleitem
    {\ml{Mobile devices programming}       {Programación de dispositivos móbiles}}
    {\ml{Graphics programming}             {Programación de gráficos}}
\cvlistdoubleitem
    {\ml{Open Source}                      {Software libre}}
    {\ml{Automation}                       {Automatización}}
\cvlistdoubleitem
    {\ml{Intelligent systems and robotics} {Sistemas Inteligente}}
    {\ml{Embedded systems programming}     {Programación de sistemas empotrados}}

%--

\section{\ml{Other interests}{Otros intereses y aficiones}}

\cvlistdoubleitem
    {\ml{Music}                      {Musica}}
    {\ml{Aeronautics}                {Aeronautica}}
\cvlistdoubleitem
    {\ml{Electronics}                {Electrónica}}
    {\ml{Puzzles/challenges}         {Puzzles y juegos de lógica}}
%\cvlistdoubleitem
%   {\ml{Sports(tennis, basketball)} {Sports(tennis, basketball)}}
%   {\ml{Pinballs}                   {Pinballs}}

\renewcommand{\listitemsymbol}{-~}            % To change the simbol for lists

\end{document}

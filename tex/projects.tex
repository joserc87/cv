\section{\ml{Projects @ \textit{Bigdata.com}. 2024 -- 2025}{Proyectos @ \textit{Bigdata.com}. 2024 -- 2025}}
\ml{
Even though \textit{Bigdata.com} is a product by \textit{RavenPack} (read
below), I think everything around it  is different enough to be considered as a
separate organization. I only entered the project one year before the public
release of \textit{Bigdata.com}, I was not part in the early stages, but I was
in charge of creating some of the systems that are now part of it.

In inverse chronological order, the main projects I've contributed to are:

\noindent\rule{\textwidth}{1pt}
}{
Aunque \textit{Bigdata.com} es un producto de \textit{RavenPack} (read below),
el tipo de trabajo es tan diferente que puede ser considerado como una
organización aparte. Entré en el proyecto solo un año antes del lanzamiento
público de \textit{Bigdata.com} así que no formé parte del proyecto desde el
principio, pero sí he estado al cargo de crear algunos de los sistemas que
componen Bigdata.

En orden descendente, los principales proyectos en los que he contribuido han
sido:

\noindent\rule{\textwidth}{1pt}
}

\cvitem{Bigdata.com - Leading Auth, Payments and Quotas}
    {Leading the team in charge of managing everything related to Auth,
    Subscriptions and Quotas.
    \textbf{Challenges}: High availability of the Auth system. Design of permissions.
    High throughput of the quota system in which the rest of the system relies
    on. Lambdas, SQS, Kinesis streams, DynamoDB, etc.
    }

\cvitem{Bigdata.com - Subscriptions and payments}
    {Lead the implementation of an integration with a 3rd party provider.
    \textbf{Challenges}: idempotency and fault tolerancy (webhooks).}

\cvitem{Bigdata.com - Python client libraries}
    {Lead the implementation of the client library for bigdata.com
    (\href{https://pypi.org/project/bigdata-client/}{bigdata-client}).
    Maintainance of our legacy client library for the RavenPack API
    (\href{https://pypi.org/project/ravenpackapi/}{ravenpackapi}).
    \textbf{Challenges}: Design the library with focus on DX. Github actions}

%----------------------------------------------------------------------------------------------------

\section{\ml{Projects @ \textit{RavenPack}. 2017 -- 2023}{Proyectos @ \textit{RavenPack}. 2017 -- 2025}}

\ml{
Hired initially as a \textit{fullstack engineer}, mostly in Python and React, I
transitioned to a more \textit{backend only} role after some years, learning
more about infrastructure (AWS). During these years I also experienced being
tech lead in many projects, being mentor to many developers, and even managed a
team of developers.

In descending order, the main projects I've contributed to are:

\noindent\rule{\textwidth}{1pt}
}{
Contratado inicialmente como \textit{ingeniero fullstack}, principalmente con
Python and React, pasé a un rol más específico de \textit{backend} a lo largo
de los años (pero aún usando Python) donde aprendí más sobre infraestructura
AWS. Durante estos años he sido lider técnico en muchos proyectos, he tenido la
oportunidad de ser mentor de algunos desarrolladores, e incluso he gestionado
un equipo como manager.\par

En orden descendente, los principales proyectos en los que he contribuido han
sido:

\noindent\rule{\textwidth}{1pt}
}

\cvitem{Data Services}
    {Maintenance of many sytems and services that are at the core of
    \textit{RavenPack}
    (API, data querying, generation (demand and historical) of archives,
    main RavenPack web, etc).
    \textbf{Challenges}: Legacy systems. ECS, Athena, ElasticSearch, DDB, CDK,
    and a big etc.}

\cvitem{Internal Tools}
    {Managing a team responsible of the maintenance of a legacy internal tool
    for managing a big knowledge base of entities. Oracle. Django. Python2.7.
    \textbf{Challenges}: Managing and mentoring a team of developers. Improving
    very old systems. Reaching a middleground between business urgent needs and
    the stability of the system.}

\cvitem{RavenPack Enterprise}
    {Fullstack development of a Web application (python + React) for searching
    Gigabytes of content. ElasticSearch, DynamoDB... \textbf{Challenges}:
    Learning Python, React, AWS, etc}

%----------------------------------------------------------------------------------------------------

\section{\ml{Projects @ \textit{The Consultancy Firm}. 2012 -- 2017}{Proyectos @ \textit{The Consultancy Firm}. 2012 -- 2017}}

\ml{
The work I did at \textit{The Consultancy Firm} was very much R\&D in nature,
which allowed me to learn many different languages and technologies, from iOS
programming to fullstack development or PC applications.

\noindent\rule{\textwidth}{1pt}
}{
El trabajo en \textit{The Consultancy Firm} fue de naturaleza I+D, lo que me
permitió aprender muchos lenguajes y tecnologicas diferentes, desde
programación en iOS como desarrollo fullstack o apps para PC.

\noindent\rule{\textwidth}{1pt}
}


%--

\cvitem{The Document Wizard}
    {\ml
    % EN
    {Our main selling project. A web application written in C\# (backend) using ASP.Net,
    Ext.Net (frontend), MicroSoft SQL Server (persistance), ActiveDirectory
    (authentication), WCF (web services), etc. This application connects to
    \textit{OpenText}'s \textit{Stream Server} via it's SOAP API to generate
    documents based on answers to questions pre-programmed by the user.}
    % ES
    {Proyecto principal. Una aplicación web escrita en C\# (backend) y usando
    ASP.Net, Ext.Net (frontend), MicroSoft SQL Server (persistencia),
    ActiveDirectory (autenticación), WCF (web services), etc. Esta aplicación
    se conecta con el producto de \textit{OpenText}, \textit{Stream Server} via
    su servicio SOAP para generar documentos PDF basados en respuestas a
    preguntas pre-programadas por el usuario.}
}

%--

\cvitem{DocWiz}
    {\ml
    % EN
    {An iOS app written in Objective-C that works as a client (online and
    offline) for \textit{The Document Wizard}.}
    % ES
    {Una app para iOS escrita en Objective-C que hace las veces de cliente
    (online u offline) para \textit{The Document Wizard}.}
}

%--

\cvitem{Augmented Documents}
    {\ml
    % EN
    {An augmented reality app for iOS using OpenGL, and MetaIO SDK. Proof of concept}
    % ES
    {Una app de realidad aumentada para iOS usando OpenGL y el SDK de MetaIO. Prueba de concepto.}
}

%--

\cvitem{The Apprentice}
    {\ml
    % EN
    {An intuitive WYSIWYG editor written in Java, with Swing for the GUI, to
    generate "Question" configurations for The Document Wizard.  Those
    configurations are marshalled to a file or directly uploaded with SOAP.}
    % ES
    {Editor WYSIWYG escrito en Java y Swing para generar archivos de
    configuracion de "Preguntas" para \textit{The Document Wizard}. Esta
    configuración es serializada (marshalled) en XML y enviada a través de
    SOAP.}
}

%--

\cvitem{Spell}
    {\ml
    % EN
    {A compiler that translates a language designed specifically for
    \textit{The Document Wizard} (spell) to XML format. This language is
    created to ease the writting of configuration files, removing the overhead
    that XML has.}
    % ES
    {Un compilador que traduce un lenguaje especificamente diseñado para
    \textit{The Document Wizard} (spell), en formato XML. Este lenguaje es
    creado con la idea de facilitar, simplificar y acortar la creación de
    archivos de configuración.}
}

%--

\cvitem{Sandd Connector}
    {\ml
    % EN
    {A \textit{StreamServe} connector written in Java to sort the letters in the
    printing queue of a printshop based on a network file with the physical
    addresses in the Netherlands and according to a sorting and classification
    algorithm, so that the letters are produced in which
    \href{https://sandd.nl/}{Sandd} would deliver them, minimizing the costs of
    posting.}
    % ES
    {Un conector para \textit{StreamServe} escrito en Java para ordenación de
    cartas de una cola de impresión de una imprenta, en base a una red con
    todas las direcciones de los países bajos y de acuerdo con unos algorimos y
    estandares de clasificación, de forma que la correspondencia es generada en
    el orden en la que el sistema de correos holandés
    (\href{https://sandd.nl/}{Sandd}) las envía, minimizando los costes.}
}

%--

\cvitem{PDF connector}
    {\ml
    % EN
    {A \textit{StreamServe connector} written in Java and using the PDFBox
    library to fix `faulty' PDF output that \textit{StreamServe} generates.}
    % ES
    {Conector para \textit{StreamServe} escrito en Java y usando la biblioteca
    PDFBox para solucionar `problemas' en los archivos PDF generatods por
    este.}
}

%--

\cvitem{PS connector}
    {\ml
    % EN
    {A \textit{StreamServe connector} written in Java that parses
    \textit{PostScript} output to fix ad-hoc problems in its fonts.}
    % ES
    {Conector para \textit{StreamServe} escrito en Java que analiza archivos
    \textit{PostScript} y soluciona ciertos `problemas' relacionados con fuentes
    de texto.}
}

%--

\cvitem{ASure}
    {\ml
    % EN
    {A notification connector in Java for error reporting to a MicroSoft SQL
    Server database using JDBC.}
    % ES
    {Conector Java para generar informes de error y rastreo en una base de
    datos Microsoft SQL Server.}
}

%--

\cvitem{SocioPath}
    {\ml
    % EN
    {A Single Page Application written in \textit{NodeJS} using the
    \textit{MEAN} stack (\textit{MongoDB, ExpressJS, Angular 1, Socket.io,
    etc}) to integrate multiple social network communication channels (Facebook
    messenger, WhatsApp, Telegram and more) for customer service. It also acts
    as a messenger bot. \textbf{Challenges}: Fist SPA, Angular, WebSockets...}
    % ES
    {Aplicación web escrita en \textit{NodeJS} usando \textit{MEAN stack}
    (\textit{MongoDB, ExpressJS, Angular 1, Socket.io, etc}) que integra
    canales de comunicación de multiples redes sociales (Facebook messenger,
    WhatsApp, Telegram y más) para facilitar el servicio al cliente). También
    actua como un text-bot para messenger. \textbf{Desafíos}: Primera SPA,
    Angular, WebSockets...}
}

%--

%\cvitem{\ml{Tools}{Herramientas}}
%    {\ml
%    {Many different small tools in Python or Java.}
%    {Diferentes herramientas pequeñas en Python o Java.}
%    }

\section{\ml{Projects @ RavenPack. 2017 -- 2024}{Proyectos @ RavenPack. 2017 -- 2024}}

\ml{
Hired initially as a \textit{fullstack engineer}, mostly in Python and React, I
transitioned to a more \textit{backend only} role, learning more about
infrastructure AWS. During these years I've also experience being tech lead of
many projects, being mentor to many developers, and I've even managing a team
of developers.

In descending order, the main projects I've worked on were (disclaimer,
obviously it's not nearly a complete list):

\noindent\rule{\textwidth}{1pt}
}{
Contratado inicialmente como \textit{ingeniero fullstack}, principalmente con
Python and React, pasé a un rol más \textit{backend} (pero aún usando Python)
donde aprendí más sobre infraestructura AWS. Durante estos años he sido lider
técnico en muchos proyectos, he sido mentor de muchos desarrolladores, e
incluso he gestionado un equipo como manager.\par

En orden descendente, los principales proyectos en los que he trabajado han
sido:

\noindent\rule{\textwidth}{1pt}
}

\cvitem{Subscriptions and payments for Bigdata.com}
    {Lead the implementation of an integration with a 3rd party provider, with
    high availability, idempotency (webhooks), etc. Lambdas, Kinesis streams,
    DynamoDB...}

\cvitem{Python client libraries}
    {Lead the implementation of the client library for bigdata.com
    (\texttt{bigdata-client} in PyPI), and maintainance of our legacy client
    library for the RavenPack API (\texttt{ravenpackapi}). Github actions}

\cvitem{Data Services}
    {Maintenance of many sytems and services that are core for the organization
    (API, data querying, file generation, main RavenPack web, etc). ECS,
    Athena, ElasticSearch, DDB, and a big etc.}

\cvitem{Internal Tools}
    {Managing a team responsible of the maintenance of a legacy internal tool
    for managing a big knowledge base of entities. Oracle. Django. Python2.7.
    \textbf{Managing a development team developers}}

\cvitem{RavenPack Enterprise}
    {Fullstack development of a Web application (python + React) for searching
    Gigabytes of content. ElasticSearch, DynamoDB...}


%--

\section{\ml{Projects @ The Consultancy Firm}{Proyectos @ The Consultancy Firm}}

\ml{
The work I did at \textit{The Consultancy Firm} was R\&D in nature, which allowed me
to learn many different languages and technologies, from iOS programming to
fullstack development or PC apps.

\noindent\rule{\textwidth}{1pt}
}{
El trabajo en \textit{The Consultancy Firm} fue de naturaleza I+D, lo que me
permitió aprender muchos lenguajes y tecnologicas diferentes, desde
programación en iOS como desarrollo fullstack o apps para PC.

\noindent\rule{\textwidth}{1pt}
}


%--

\cvitem{The Document Wizard}
    {\ml
    % EN
    {Main project. A web application written in C\# (backend) using ASP.Net,
    Ext.Net (frontend), MicroSoft SQL Server (persistance), ActiveDirectory
    (authentication), WCF (web services), etc. This application connects to
    \textit{OpenText}'s \textit{Stream Server} via it's SOAP API to generate
    documents based on answers to questions pre-programmed by the user.}
    % ES
    {Proyecto principal. Una aplicación web escrita en C\# (backend) y usando
    ASP.Net, Ext.Net (frontend), MicroSoft SQL Server (persistencia),
    ActiveDirectory (autenticación), WCF (web services), etc. Esta aplicación
    se conecta con el producto de \textit{OpenText}, \textit{Stream Server} via
    su servicio SOAP para generar documentos PDF basados en respuestas a
    preguntas pre-programadas por el usuario.}
}

%--

\cvitem{DocWiz}
    {\ml
    % EN
    {An iOS app written in Objective-C that works as an online or offline client
    for \textit{The Document Wizard}.}
    % ES
    {Una app para iOS escrita en Objective-C que hace las veces de cliente
    (online u offline) para \textit{The Document Wizard}.}
}

%--

\cvitem{Augmented Documents}
    {\ml
    % EN
    {An augmented reality app for iOS using OpenGL, and MetaIO SDK. Proof of concept}
    % ES
    {Una app de realidad aumentada para iOS usando OpenGL y el SDK de MetaIO. Prueba de concepto.}
}

%--

\cvitem{The Apprentice}
    {\ml
    % EN
    {An intuitive WYSIWYG editor written in Java, with Swing for the GUI, to
    generate "Question" configurations for The Document Wizard.  Those
    configurations are marshalled to a file or directly uploaded with SOAP.}
    % ES
    {Editor WYSIWYG escrito en Java y Swing para generar archivos de
    configuracion de "Preguntas" para \textit{The Document Wizard}. Esta
    configuración es serializada (marshalled) en XML y enviada a través de
    SOAP.}
}

%--

\cvitem{Spell}
    {\ml
    % EN
    {A compiler that translates a language designed specifically for
    \textit{The Document Wizard} (spell) to XML format. This language is
    created to ease the writting of configuration files, removing the overhead
    that XML has.}
    % ES
    {Un compilador que traduce un lenguaje especificamente diseñado para
    \textit{The Document Wizard} (spell), en formato XML. Este lenguaje es
    creado con la idea de facilitar, simplificar y acortar la creación de
    archivos de configuración.}
}

%--

\cvitem{Sandd Connector}
    {\ml
    % EN
    {A \textit{StreamServe} connector written in Java to sort the letters in the
    printing queue of a printshop based on a network file with the physical
    addresses in the Netherlands and according to a sorting and classification
    algorithm, so that the letters are produced in which Sandd would deliver
    them, minimizing the costs of posting.}
    % ES
    {Un conector para \textit{StreamServe} escrito en Java para ordenación de
    cartas de una cola de impresión de una imprenta, en base a una red con
    todas las direcciones de los países bajos y de acuerdo con unos algorimos y
    estandares de clasificación, de forma que la correspondencia es generada en
    el orden en la que el sistema de correos holandés (Sandd) las envía,
    minimizando los costes.}
}

%--

\cvitem{PDF connector}
    {\ml
    % EN
    {A \textit{StreamServe connector} written in Java and using the PDFBox
    library to fix `faulty' PDF output that \textit{StreamServe} generates.}
    % ES
    {Conector para \textit{StreamServe} escrito en Java y usando la biblioteca
    PDFBox para solucionar `problemas' en los archivos PDF generatods por
    este.}
}

%--

\cvitem{PS connector}
    {\ml
    % EN
    {A \textit{StreamServe connector} written in Java that parses
    \textit{PostScript} output to fix problems in its fonts.}
    % ES
    {Conector para \textit{StreamServe} escrito en Java que analiza archivos
    \textit{PostScript} y soluciona ciertos `problemas' relacionados con fuents
    de texto.}
}

%--

\cvitem{ASure}
    {\ml
    % EN
    {A notification connector in Java for error reporting to a MicroSoft SQL
    Server database using JDBC.}
    % ES
    {Conector Java para generar informes de error y rastreo en una base de
    datos Microsoft SQL Server.}
}

%--

\cvitem{SocioPath}
    {\ml
    % EN
    {A reactive web application written in \textit{NodeJS} using the
    \textit{MEAN} stack (\textit{MongoDB, ExpressJS, Angular 1, Socket.io,
    etc}) to integrate multiple social network communication channels (Facebook
    messenger, WhatsApp, Telegram and more) for customer service. It also acts
    as a messenger bot.}
    % ES
    {Aplicación web escrita en \textit{NodeJS} usando \textit{MEAN stack}
    (\textit{MongoDB, ExpressJS, Angular 1, Socket.io, etc}) que integra
    canales de comunicación de multiples redes sociales (Facebook messenger,
    WhatsApp, Telegram y más) para facilitar el servicio al cliente). También
    actua como un text-bot para messenger.}
}

%--

%\cvitem{\ml{Tools}{Herramientas}}
%    {\ml
%    {Many different small tools in Python or Java.}
%    {Diferentes herramientas pequeñas en Python o Java.}
%    }
